%!TEX root = ../template.tex
%%%%%%%%%%%%%%%%%%%%%%%%%%%%%%%%%%%%%%%%%%%%%%%%%%%%%%%%%%%%%%%%%%%%
%% chapter4.tex
%% NOVA thesis document file
%%%%%%%%%%%%%%%%%%%%%%%%%%%%%%%%%%%%%%%%%%%%%%%%%%%%%%%%%%%%%%%%%%%%

\typeout{NT FILE chapter4.tex}%

\chapter{Solução Proposta}\label{cha:solucao_proposta}

Este capítulo apresenta a solução proposta para a integração de técnicas de inteligência artificial no processo de revisão de código da Processware. É descrita a abordagem adotada, a arquitetura geral do sistema e os seus principais componentes, bem como a forma como a solução se integra nos processos internos e nas ferramentas já utilizadas pela organização.

Ao contrário de abordagens reativas baseadas em eventos da plataforma de controlo de versões, a solução proposta assenta num modelo proativo, no qual a análise das alterações de código é desencadeada antes da criação de uma \gls{PR}. Esta decisão permite alinhar a solução com as práticas existentes na Processware e reforçar o papel da inteligência artificial como mecanismo de apoio à qualidade do código desde fases iniciais do desenvolvimento.

\section{Visão Geral}

A solução proposta consiste numa aplicação externa integrada com o software interno da Processware, o O2P, concebida para apoiar a revisão de código através da análise automática de diferenças entre ramos de desenvolvimento. O sistema permite comparar uma \textit{branch} de trabalho com a respetiva \textit{branch} base, identificando alterações relevantes e gerando comentários de revisão alinhados com as boas práticas internas da organização.

O sistema é acionado pelo programador no O2P, numa fase anterior à criação do \gls{PR}. Após a solicitação, a aplicação procede à extração e ao pré-processamento das diferenças entre os ramos selecionados, organizando a informação de forma adequada para dar início à análise das alterações feitas pelo programador.

As alterações processadas são analisadas por um \textit{software} baseado em \glspl{LLM}, que combina o código alterado com conhecimento derivado de boas práticas internas e de revisões anteriores. Este mecanismo permite gerar comentários de revisão em linguagem natural, identificando potenciais problemas, desvios a padrões definidos e oportunidades de melhoria.

Os comentários gerados são apresentados ao programador no O2P, permitindo a sua análise e incorporação antes da submissão formal da \gls{PR}. Desta forma, o sistema atua como um mecanismo preventivo de qualidade, complementando as ferramentas de análise estática existentes e preservando o papel central da revisão final realizada pelos respetivos revisores de código.

\section{Processamento e Utilização das Boas Práticas Internas}\label{sec:boas_praticas}

As boas práticas internas utilizadas pelo motor de análise não resultam de um processo de aprendizagem automática ou de treino de modelos sobre dados históricos. Em vez disso, estas práticas correspondem a um conjunto de regras, convenções e recomendações já existentes na organização, atualmente documentadas em guias internos, normas de desenvolvimento e conhecimento partilhado entre equipas.

No contexto da solução proposta, estas boas práticas são formalizadas e estruturadas de forma a poderem ser utilizadas como conhecimento de apoio durante o processo de análise automática. A sua integração no sistema é realizada através da disponibilização explícita deste conhecimento ao motor de análise baseado em \glspl{LLM}, sob a forma de contexto adicional que orienta a geração de comentários de revisão.

Desta forma, o modelo de linguagem não é treinado com dados específicos da organização, mas sim utilizado como um mecanismo de interpretação e raciocínio sobre as alterações de código, tendo como referência um conjunto de boas práticas previamente definidas. Esta abordagem permite beneficiar das capacidades dos \glspl{LLM} sem incorrer nos custos, riscos e complexidades associados ao treino ou ajuste fino de modelos em ambientes empresariais.

A base de conhecimento de boas práticas pode ser atualizada de forma independente, refletindo a evolução das normas internas da Processware, sem necessidade de alterações ao motor de análise ou à arquitetura global da solução.

\section{Requisitos e Condições de Utilização}\label{sec:requisitos_condicoes}

Estes elementos servem de base ao desenho da arquitetura e asseguram que a solução proposta se integra de forma consistente nos processos internos da organização.

\subsection{Requisitos Funcionais}

Os requisitos funcionais definem as capacidades que o sistema deve oferecer para apoiar o processo de revisão de código:

\begin{itemize}
    \item \textbf{RF1 -- Seleção de ramos para análise:}  
    O sistema deve permitir ao utilizador selecionar a \textit{branch} de trabalho e a \textit{branch} base a comparar.

    \item \textbf{RF2 -- Extração automática de diferenças:}  
    O sistema deve extrair programaticamente as diferenças entre os ramos selecionados, incluindo o \gls{Diff} e os ficheiros modificados.

    \item \textbf{RF3 -- Pré-processamento das alterações:}  
    O sistema deve filtrar e organizar as alterações de código de forma a otimizar a análise automática.

    \item \textbf{RF4 -- Análise baseada em inteligência artificial:}  
    O sistema deve analisar as alterações utilizando \glspl{LLM} para identificar problemas e oportunidades de melhoria.

    \item \textbf{RF5 -- Integração de boas práticas internas:}  
    A análise deve ser guiada por um conjunto formalizado de boas práticas da Processware.

    \item \textbf{RF6 -- Geração de comentários de revisão:}  
    O sistema deve gerar comentários claros e concisos em linguagem natural.

    \item \textbf{RF7 -- Apresentação dos resultados no O2P:}  
    Os comentários devem ser apresentados ao utilizador no O2P antes da criação da \gls{PR}.
\end{itemize}

\subsection{Requisitos Não Funcionais}

A solução deve igualmente cumprir um conjunto de requisitos não funcionais:

\begin{itemize}
    \item \textbf{RNF1 -- Integração com sistemas internos:}  
    A solução deve integrar-se de forma nativa com o O2P.

    \item \textbf{RNF2 -- Baixo acoplamento a plataformas externas:}  
    O sistema não deve depender diretamente de funcionalidades específicas do Bitbucket.

    \item \textbf{RNF3 -- Escalabilidade:}  
    O sistema deve suportar múltiplas análises em paralelo.

    \item \textbf{RNF4 -- Desempenho aceitável:}  
    O tempo de resposta deve ser compatível com o fluxo de trabalho do programador.

    \item \textbf{RNF5 -- Manutenibilidade e extensibilidade:}  
    A arquitetura deve permitir a evolução futura da solução.
\end{itemize}

\subsection{Condições de Utilização}

O funcionamento do sistema pressupõe as seguintes condições:

\begin{itemize}
    \item \textbf{CU1 -- Utilização de controlo de versões baseado em ramos;}
    \item \textbf{CU2 -- Existência de boas práticas internas documentadas;}
    \item \textbf{CU3 -- Supervisão humana das recomendações;}
    \item \textbf{CU4 -- Utilização do O2P como ponto central do fluxo de desenvolvimento;}
\end{itemize}

\section{Arquitetura Geral da Solução}\label{sec:arquitetura_geral}

A solução proposta adota uma arquitetura baseada numa aplicação externa integrada no ecossistema do O2P. Esta aplicação funciona como um serviço autónomo, responsável pela análise de diferenças entre ramos de código e pela geração de comentários de revisão, não dependendo diretamente de eventos ou mecanismos internos da plataforma de controlo de versões.

Esta abordagem permite alinhar a solução com as práticas de desenvolvimento já adotadas pela Processware, privilegiando a separação de responsabilidades, a reutilização de componentes existentes e a integração com sistemas corporativos internos.

\subsection{Visão Geral da Arquitetura}

A arquitetura é composta pelos seguintes elementos principais:

\begin{itemize}
    \item O2P, enquanto sistema corporativo existente, responsável pela interação com o utilizador e pela orquestração do processo de análise;
    \item Aplicação externa de análise de código, responsável pela execução da lógica de análise automática e pela coordenação dos componentes de inteligência artificial;
    \item Motor de análise baseado em \glspl{LLM};
    \item Base de conhecimento de boas práticas internas;
    \item Sistema de controlo de versões.
\end{itemize}

\subsection{Integração com o O2P}

O O2P atua como ponto de entrada da solução, permitindo ao programador solicitar explicitamente a análise das alterações de código. A interação ocorre através do O2P, que invoca a aplicação externa de análise por meio de interfaces bem definidas.

Desta forma, o utilizador permanece no contexto da ferramenta corporativa já utilizada no seu dia a dia, enquanto a complexidade associada à análise automática baseada em inteligência artificial é abstraída e encapsulada num serviço dedicado.

\subsection{Aplicação Externa de Análise}

A aplicação externa constitui o núcleo funcional da solução proposta, sendo responsável por extrair as diferenças entre ramos de código a partir do sistema de controlo de versões, executar o pré-processamento das alterações e orquestrar a análise baseada em inteligência artificial.

Após a execução do processo de análise, os resultados são devolvidos ao O2P, que se encarrega da sua apresentação ao programador. Esta abordagem reduz o acoplamento com ferramentas externas específicase e facilita a evolução futura da solução.

\subsection{Motor de Análise e Base de Conhecimento}

O motor de análise combina as alterações de código com um conjunto de boas práticas internas previamente formalizadas, permitindo a geração de comentários de revisão contextualizados e alinhados com os padrões da organização.

A base de conhecimento é mantida de forma independente do motor de análise, possibilitando a sua atualização contínua sem impacto direto na lógica do sistema. Esta separação facilita a adaptação da solução à evolução das práticas internas da Processware e contribui para a sua manutenção a longo prazo.




\section{Desenho Detalhado dos Componentes}\label{sec:desenho_componentes}

Esta secção descreve de forma detalhada os principais componentes da solução proposta, bem como as respetivas responsabilidades e interações. O objetivo é clarificar a decomposição interna do sistema, evidenciando como cada componente contribui para o fluxo de execução apresentado na Secção.

A solução foi desenhada segundo princípios de modularidade e separação de responsabilidades, permitindo facilitar a manutenção, a evolução futura e a reutilização de componentes em diferentes contextos.

\subsection{Componente de Integração com o Sistema de Controlo de Versões}

O componente de integração com o sistema de controlo de versões é responsável pela interação entre a aplicação externa de análise e os repositórios de código utilizados pela Processware. Este componente permite obter programaticamente o conteúdo dos ramos a comparar, bem como extrair as diferenças de código relevantes para análise.

Ao contrário de abordagens baseadas em eventos ou mecanismos reativos da plataforma de controlo de versões, este componente é acionado explicitamente pela aplicação externa, no contexto de um pedido iniciado a partir do O2P. Esta opção reduz a dependência de funcionalidades específicas da plataforma e permite alinhar o processo de análise com os fluxos internos da Processware.

\subsection{Componente de Gestão do Processo de Análise}

O componente de gestão do processo de análise é responsável por coordenar a execução do fluxo de análise após a receção de um pedido válido proveniente do O2P. Este componente determina se uma análise deve ser executada, com base em critérios previamente definidos, como o tipo de projeto, o estado do código ou as tecnologias envolvidas.

Ao centralizar esta lógica, o sistema garante um controlo consistente sobre quando e como a análise automática é realizada, evitando execuções desnecessárias e contribuindo para a eficiência global da solução.

\subsection{Componente de Pré-processamento}

O componente de pré-processamento tem como principal função preparar as alterações de código para análise automática. Este componente aplica operações como a filtragem de ficheiros irrelevantes, a normalização do formato das diferenças entre ramos e a identificação das linguagens de programação envolvidas.

Este passo é particularmente relevante em projetos empresariais complexos, onde coexistem múltiplas tecnologias e diferentes estilos de codificação. O pré-processamento contribui para melhorar a qualidade da análise subsequente e para reduzir ruído nos resultados gerados pelo sistema.

\subsection{Componente de Organização da Análise}

O componente de organização da análise é responsável por coordenar a interação entre os dados pré-processados, a base de conhecimento de boas práticas internas e o motor de análise baseado em \glspl{LLM}. Este componente constrói o contexto de análise, garantindo que a informação relevante é corretamente estruturada e apresentada ao motor de IA.

Para além disso, este componente gere aspetos como a segmentação das alterações em unidades analisáveis, o controlo do volume de informação enviado ao modelo e a consolidação dos resultados obtidos. Esta responsabilidade é essencial para assegurar uma utilização eficiente e controlada dos \glspl{LLM}.

\subsection{Motor de Análise Baseado em LLMs}

O motor de análise constitui o núcleo inteligente da solução proposta. Este componente utiliza \glspl{LLM} para interpretar as alterações de código de forma contextual, indo além da análise sintática tradicional. Através da combinação entre o código alterado e as boas práticas internas da Processware, o motor é capaz de gerar comentários de revisão em linguagem natural.

Apesar das capacidades avançadas deste componente, o sistema irá ser desenhado de forma a não depender exclusivamente dos resultados produzidos pelo modelo, reconhecendo as limitações associadas à utilização de IA generativa em contextos críticos.

\subsection{Componente de Validação e Filtragem}

O componente de validação e filtragem tem como objetivo garantir a qualidade dos comentários gerados pelo motor de análise. Este componente aplica regras adicionais para eliminar comentários redundantes, identificar possíveis falsos positivos e priorizar feedback com maior relevância para os revisores humanos.

\subsection{Base de Conhecimento de Boas Práticas Internas}

A base de conhecimento de boas práticas internas armazena um conjunto formalizado de regras, convenções e recomendações específicas da Processware. Este componente permite capturar conhecimento organizacional que, de outra forma, estaria disperso ou implícito na experiência dos programadores mais experientes.

A separação desta base de conhecimento do motor de análise permite a sua evolução independente, facilitando a atualização das boas práticas sem necessidade de alterações significativas na lógica do sistema.

\subsection{Componente de Disponibilização de Resultados}

O componente de disponibilização de resultados é responsável por devolver ao O2P os comentários de revisão gerados pelo sistema, permitindo a sua apresentação ao utilizador no contexto da ferramenta corporativa.

Este componente assegura que o feedback produzido pela análise automática é estruturado de forma clara e consistente, facilitando a sua interpretação e utilização como apoio ao processo durante o processo de revisão de código.

\subsection{Síntese do Desenho de Componentes}

A decomposição da solução em componentes bem definidos permite satisfazer os requisitos identificados na Secção~\ref{sec:requisitos_condicoes}, promovendo a modularidade, a manutenibilidade e a extensibilidade do sistema. Esta abordagem facilita também a transição para a fase de implementação, descrita nos capítulos seguintes, onde cada componente poderá ser mapeado para tecnologias e ferramentas concretas utilizadas pela Processware.

\section{Considerações de Implementação e Decisões Técnicas}\label{sec:consideracoes_implementacao}

A adoção de uma aplicação externa integrada no O2P, em detrimento de uma integração direta com o Bitbucket, reflete a preocupação em alinhar a solução com as práticas internas da Processware. Esta decisão reduz dependências externas, reforça o controlo do fluxo de análise e permite uma integração mais natural da inteligência artificial no processo de desenvolvimento.

A solução complementa ferramentas existentes, como o SonarLint, e preserva o papel central do julgamento humano, assegurando que a inteligência artificial atua como um mecanismo de apoio e não como um substituto do processo de revisão tradicional.

\section{Plano de trabalho}

Esta secção apresenta o plano de trabalho definido para a implementação e validação da solução proposta. O plano encontra-se estruturado em várias fases, organizadas de forma a assegurar uma progressão consistente entre a extração e processamento de dados, o desenvolvimento do sistema e a sua validação final.

O planeamento das atividades teve em consideração a natureza incremental da solução, bem como a necessidade de integrar componentes de inteligência artificial de forma controlada e alinhada com os processos internos da organização. O diagrama de Gantt apresentado na Figura~\ref{fig:gantt} ilustra a distribuição temporal das principais fases do projeto, evidenciando a sobreposição de tarefas sempre que adequado, de modo a otimizar o tempo de desenvolvimento.

\begin{figure}[h]
    \centering
    \includegraphics[width=\textwidth]{5-Figures/Thesis_Images/gantt}
    \caption{Diagrama de Gantt do plano de trabalho}
    \label{fig:gantt}
\end{figure}

O plano de trabalho encontra-se organizado em várias fases principais, cada uma correspondendo a um conjunto de atividades com objetivos bem definidos:

\subsection*{Extração e Processamento de Dados}
Esta fase é dedicada à implementação dos mecanismos de acesso ao sistema de controlo de versões e à extração das alterações de código relevantes para análise. Inclui igualmente a organização e estruturação das boas práticas internas da organização, de forma a possibilitar a sua utilização como conhecimento de apoio durante a análise automática.

\subsection*{Infraestrutura do LLM}
Nesta fase é definida e configurada a infraestrutura necessária para a execução do motor de análise baseado em modelos de linguagem de grande escala, assegurando requisitos de desempenho, fiabilidade e controlo no acesso ao modelo.

\subsection*{Implementação do Sistema}
A fase de implementação do sistema contempla o desenvolvimento da lógica principal da solução, incluindo a comparação entre \textit{branches}, o pré-processamento das alterações de código, a integração das boas práticas internas e a geração de comentários de revisão baseados em inteligência artificial. Adicionalmente, esta fase inclui a implementação da interface de integração com o O2P, garantindo que os resultados da análise são apresentados de forma clara e acessível aos programadores.

\subsection*{Desenvolvimento da Dissertação}
Em paralelo com o desenvolvimento técnico, é realizada a redação do capítulo de implementação da dissertação, documentando as decisões técnicas adotadas e a arquitetura efetivamente implementada.

\subsection*{Integração e Validações}
A fase final é dedicada à integração dos diferentes componentes do sistema no processo de revisão de código da Processware e à realização de um conjunto de validações, com o objetivo de assegurar o comportamento esperado da solução e a qualidade do feedback gerado antes da conclusão do trabalho.