%!TEX root = ../template.tex
%%%%%%%%%%%%%%%%%%%%%%%%%%%%%%%%%%%%%%%%%%%%%%%%%%%%%%%%%%%%%%%%%%%
%% chapter1.tex
%% NOVA thesis document file
%%
%% Chapter with introduction
%%%%%%%%%%%%%%%%%%%%%%%%%%%%%%%%%%%%%%%%%%%%%%%%%%%%%%%%%%%%%%%%%%%

\typeout{NT FILE chapter1.tex}%

\chapter{Introdução}\label{cha:introdução}

Este capítulo tem como objetivo introduzir o tema e objetivos desta dissertação. É apresentada a motivação para a realização do projeto, o contexto em que o problema abordado se insere e os objetivos a serem alcançados.

\section{Motivação}

O aumento da complexidade do software moderno e o ritmo acelerado de desenvolvimento tornam a Revisão de Código uma etapa essencial no ciclo de vida do software. Este processo garante a qualidade, segurança e conformidade do código com as boas práticas de engenharia e com os padrões internos definidos pelas organizações~\cite{tufano2024slr,llmsurvey2023}.

Contudo, a revisão de código continua a ser uma tarefa exigente e demorada, que frequentemente exige múltiplos revisores para evitar a introdução de erros~\cite{liang2024automation}. Em ambientes empresariais com grande volume de commits e pull requests, os revisores lidam diariamente com alterações extensas que exigem análise cuidadosa. Isto pode originar atrasos, inconsistências e risco de falhas quando as mudanças chegam ao produto final~\cite{rahman2024empirical,llmsurvey2023}.

No contexto específico desta dissertação, estes desafios tornam-se evidentes. A Processware mantém uma base de código ampla, distribuída por vários repositórios e com múltiplas tecnologias — como C, C++, C\#, Vue, SQL — coexistindo com código legacy. Esta diversidade aumenta a complexidade da revisão e exige conhecimentos técnicos específicos por parte dos revisores~\cite{zhong2024evaluation}. Além disso, diferentes equipas podem aplicar práticas distintas, dificultando a uniformização de padrões internos e levando a revisões repetitivas e demoradas~\cite{liang2024automation}.

Com o avanço recente dos Modelos de Linguagem (LLMs), já amplamente demonstrado em ferramentas como ChatGPT, Gemini ou GitHub Copilot, tornou-se evidente que estes modelos têm capacidade para compreender e analisar código-fonte, identificar problemas e sugerir melhorias~\cite{llmsurvey2023,nguyen2025comparison}. Assim, a adoção de LLMs na revisão de código surge como uma oportunidade para reduzir esforço manual, aumentar a consistência e melhorar a eficiência do processo~\cite{rahman2024empirical,zhong2024evaluation}.

\section{Contexto}

Esta dissertação enquadra-se no conjunto de soluções desenvolvidas pela Processware para melhorar o processo interno de revisão de código.

Atualmente, a revisão de código na empresa é totalmente manual e depende fortemente da experiência dos revisores. O elevado volume de alterações, a variedade tecnológica e a coexistência de código moderno com código legacy tornam a revisão uma tarefa exigente e pouco escalável~\cite{rahman2024empirical}. Esta situação resulta em tempos de análise elevados e inconsistências entre equipas, dificultando a padronização de práticas internas~\cite{zhong2024evaluation}.

Dado este cenário, torna-se pertinente explorar soluções baseadas em inteligência artificial, capazes de apoiar a análise de código e contribuir para um processo de revisão mais eficiente, coerente e alinhado com as necessidades atuais de desenvolvimento de software. Estudos recentes mostram que LLMs podem atuar como revisores assistentes, melhorar a qualidade dos comentários, aumentar a uniformidade e acelerar a deteção de problemas~\cite{tufano2024slr,autoreviewer2025fieldstudy}.

\section{Definição do Problema}\label{sec:definicao_do_problema}



\section{Objetivos}\label{sec:objetivos}



\section{Solução Proposta}\label{sec:solucao_proposta}


