%!TEX root = ../template.tex
%%%%%%%%%%%%%%%%%%%%%%%%%%%%%%%%%%%%%%%%%%%%%%%%%%%%%%%%%%%%%%%%%%%
%% chapter1.tex
%% NOVA thesis document file
%%
%% Chapter with introduction
%%%%%%%%%%%%%%%%%%%%%%%%%%%%%%%%%%%%%%%%%%%%%%%%%%%%%%%%%%%%%%%%%%%

\typeout{NT FILE chapter1.tex}%

\chapter{Introdução}\label{cha:introdução}

Este capítulo tem como objetivo introduzir o tema e objetivos desta dissertação. É apresentada a motivação para a realização do projeto, o contexto em que o problema abordado se insere e os objetivos a serem alcançados.

\section{Motivação}

Nos dias de hoje, com a crescente complexidade de software moderno e ritmo acelerado de
desenvolvimento, a etapa de Revisão de Código é cada vez uma etapa mais crítica no ciclo
de desenvolvimento de software~\cite{Tufano2024}\cite{Chen2023Survey}.
Este processo consiste em assegurar a segurança e qualidade do código desenvolvido, bem como garantir que o mesmo esteja em conformidade
com não só as boas práticas de engenharia mas também as boas práticas definidas pela
empresa, dona do software desenvolvido~\cite{Rahman2024LLMReview}.

No entanto, este processo é bastante exigente, longo e, normalmente, tem que ser realizado
por mais do que uma pessoa para que se evite ao máximo erros no final deste processo~\cite{Liang2024}.
Em projetos empresariais, onde são submetidos inúmeros commits e pull requests diariamente,
as equipas responsáveis pela revisão de código lidam com bastantes alterações que necessitam de uma análise cuidadosa
para que não existam erros quando estas mudanças chegam ao produto oficial a ser desenvolvido,
o que pode resultar em atrasos nas entregas, inconsistências e risco de existirem problemas
ao nível de produto~\cite{Rahman2024LLMReview}\cite{Liang2024}.

No contexto onde esta tese será aplicada, os desafios de Code Review tornam-se evidentes.
A Processware possui uma base de código extensa, com diversos repositórios, dependendo das necessidades,
onde são utilizadas diversas tecnologias, nomeadamente, C, C++, C\#, Vue + Vite, SQL, entre outras,
sendo estas geridas através do Bitbucket, uma aplicação de source control~\cite{Nguyen2025Comparison}.
A diversidade tecnológica em conjunto com a atual existência de código legacy (código que irá ser deprecado no futuro),
exige algumas competências técnicas por parte dos membros da equipa responsável pela revisão de código~\cite{Zhong2024Evaluation}.

Adicionalmente, devido à existência de várias equipas diferentes dentro da empresa faz com que algumas sugestões poderão não ser consistentes,
dificultando a uniformização de práticas internas e padrões. Com isto, o tempo investido pela equipa de Code Review em
alterações possivelmente específicas bem como revisões repetitivas são bastante custosas para a Processware,
desviando a atenção de tarefas mais relevantes~\cite{Liang2024}.

Atualmente, a inteligência artificial têm vindo a crescer exponencialmente através de modelos de grande escala (LLMs),
sendo bastante visível em algumas ferramentas como são os casos do ChatGPT, Gemini, Copilot e DeepSeek
que demonstram uma elevada capacidade em não só compreender código desenvolvido em diversas línguas de programação
mas também a gerar sugestões para problemas que lhes sejam apresentados~\cite{Chen2023Survey}\cite{Nguyen2025Comparison}.

Assim sendo, a utilização de modelos LLMs na etapa de revisão de código é um passo importante
para resolver os principais problemas associados a este processo~\cite{Liang2024}\cite{Zhong2024Evaluation}.

\section{Contexto}

Esta dissertação insere-se no contexto das soluções desenvolvidas pela Processware,
com o sentido de melhorar a etapa de revisão de código presente no ciclo de vida de
desenvolvimento de software.

Atualmente, a Processware não utiliza qualquer tecnologia de Inteligência Artificial
nos seus processos internos, incluindo na etapa de revisão de código, que continua a ser
totalmente manual~\cite{Liang2024}. A diversidade tecnológica, o volume de alterações submetidas
e a coexistência de código moderno e legacy tornam o processo de revisão de código exigente,
demorado e dependente da experiência individual dos revisores~\cite{Rahman2024LLMReview}\cite{Zhong2024Evaluation}.

Assim, este trabalho advém da necessidade de explorar soluções baseadas em inteligência artificial
que possam apoiar a análise de código e contribuir para um processo de revisão mais eficiente,
consistente e alinhado com as exigências atuais relativamente a desenvolvimento de software~\cite{Tufano2024}\cite{Nguyen2025Comparison}.

\section{Definição do Problema}\label{sec:definicao_do_problema}



\section{Objetivos}\label{sec:objetivos}



\section{Solução Proposta}\label{sec:solucao_proposta}


