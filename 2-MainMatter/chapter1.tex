%!TEX root = ../template.tex
%%%%%%%%%%%%%%%%%%%%%%%%%%%%%%%%%%%%%%%%%%%%%%%%%%%%%%%%%%%%%%%%%%%
%% chapter1.tex
%% NOVA thesis document file
%%
%% Chapter with introduction
%%%%%%%%%%%%%%%%%%%%%%%%%%%%%%%%%%%%%%%%%%%%%%%%%%%%%%%%%%%%%%%%%%%

\typeout{NT FILE chapter1.tex}%

\chapter{Introdução}\label{cha:introdução}

Este capítulo tem como objetivo introduzir o tema e objetivos desta dissertação. É apresentada a motivação para a realização do projeto, o contexto em que o problema abordado se insere e os objetivos a serem alcançados.

\section{Motivação}

O aumento da complexidade do software moderno e o ritmo acelerado de desenvolvimento tornam a \gls{CR} uma etapa essencial no ciclo de vida do software. Este processo garante a qualidade, segurança e conformidade do código com as boas práticas de engenharia e com os padrões internos definidos pelas organizações~\cite{tufano2024slr,llmsurvey2023}.

Contudo, a \gls{CR} continua a ser uma tarefa exigente e demorada, que frequentemente exige múltiplos revisores para evitar a introdução de erros~\cite{liang2024automation}. Em ambientes empresariais com grande volume de commits e \glspl{PR}, os revisores lidam diariamente com alterações extensas que exigem análise cuidadosa. Isto pode originar atrasos, inconsistências e risco de falhas quando as mudanças chegam ao produto final~\cite{llmsurvey2023}.

No contexto específico desta dissertação, estes desafios tornam-se evidentes. A Processware mantém uma base de código ampla, distribuída por vários repositórios e com múltiplas tecnologias — como C, C++, C\#, Vue, SQL — coexistindo com código legacy. Esta diversidade aumenta a complexidade da revisão e exige conhecimentos técnicos específicos por parte dos revisores~\cite{jiang2023codereview}. Além disso, diferentes equipas podem aplicar práticas distintas, dificultando a uniformização de padrões internos e levando a revisões repetitivas e demoradas~\cite{liang2024automation}.

Com o avanço recente dos \glspl{LLM}, já amplamente demonstrado em ferramentas como ChatGPT, Gemini ou GitHub Copilot, tornou-se evidente que estes modelos têm capacidade para compreender e analisar código-fonte, identificar problemas e sugerir melhorias~\cite{llmsurvey2023}. Assim, a adoção de \glspl{LLM} na revisão de código surge como uma oportunidade para reduzir esforço manual, aumentar a consistência e melhorar a eficiência do processo.

\section{Contexto}

Esta dissertação enquadra-se no conjunto de soluções desenvolvidas pela Processware para melhorar o processo interno de revisão de código.

A Processware é uma empresa tecnológica com presença internacional, especializada no desenvolvimento de soluções para otimização de processos e gestão operacional em ambientes empresariais complexos. A sua atividade centra-se na transformação digital de operações através da combinação de engenharia de processos e plataformas tecnológicas proprietárias, apoiando organizações na melhoria da eficiência, controlo e escalabilidade das suas operações.

No centro da sua oferta encontra-se a plataforma O2P, uma solução baseada na cloud que permite a digitalização, monitorização e otimização de processos operacionais em tempo real, integrando-se com sistemas existentes e suportando diferentes contextos tecnológicos. A empresa atua em sectores com elevada complexidade operacional, caracterizados por múltiplas tecnologias, elevados volumes de dados e necessidade de elevada fiabilidade.

Neste contexto, a Processware mantém uma base de código extensa e heterogénea, distribuída por vários repositórios e tecnologias, o que coloca desafios significativos ao nível da manutenção e da revisão de código. A necessidade de garantir qualidade, consistência e alinhamento com boas práticas internas torna o processo de \gls{CR} um elemento crítico no seu ciclo de desenvolvimento de software.

Atualmente, o processo de \gls{CR} é totalmente manual e depende fortemente da experiência dos revisores. O elevado volume de alterações, a variedade tecnológica e a coexistência de código moderno com código legacy tornam a revisão uma tarefa exigente e pouco escalável. Esta situação resulta em tempos de análise elevados e inconsistências entre equipas, dificultando a padronização de práticas internas~\cite{jiang2023codereview}.

Dado este cenário, torna-se pertinente explorar soluções baseadas em inteligência artificial, capazes de apoiar a análise de código e contribuir para um processo de \gls{CR} mais eficiente, coerente e alinhado com as necessidades atuais de desenvolvimento de software. Estudos recentes mostram que \glspl{LLM} podem atuar como revisores assistentes, melhorar a qualidade dos comentários, aumentar a uniformidade e acelerar a deteção de problemas~\cite{tufano2024slr}.

\section{Definição do Problema}\label{sec:definicao_do_problema}

O processo de \gls{CR} na Processware é atualmente manual, exigindo que os revisores analisem cada alteração submetida nos diferentes repositórios e tecnologias utilizadas pela empresa. Consequentemente, esta abordagem apresenta várias limitações. Numa primeira instância, a revisão é uma tarefa demorada e dependente das competências técnicas dos revisores, o que pode levar ao aumento da probabilidade de ocorrerem erros e inconsistências~\cite{liang2024automation}. A coexistência de múltiplas tecnologias e de código legacy também dificultam na identificação de problemas numa forma sistemática~\cite{tufano2024slr}.

Além disso, revisores poderão adotar critérios distintos, dependendo da \gls{PR} em questão, resultando em comentários heterogéneos e dificultando a uniformização de práticas internas. Estudos recentes mostram que esta falta de consistência é um problema comum em ambientes empresariais e afeta tanto a qualidade das revisões de código como a eficiência deste processo~\cite{jiang2023codereview}. Outro problema recorrente é a necessidade de realizar comentários repetitivos, como correções de estilo, validações simples ou alertas sobre riscos conhecidos, que consomem tempo e poderiam ser automatizados~\cite{liang2024automation}.

Com os avanços recentes em \glspl{LLM}, surgiu a oportunidade de automatizar grande parte deste processo. Estes modelos têm demonstrado capacidade para analisar código, identificar problemas e gerar comentários úteis~\cite{airesults2024review}. No entanto, a utilidade dos comentários varia significativamente sendo que alguns modelos tendem a gerar falsos positivos e a eficácia depende da clareza e pertinência das recomendações produzidas. Importa ainda referir que \glspl{LLM} genéricos não compreendem as práticas e padrões internos de cada organização, reduzindo a sua aplicabilidade prática durante o processo de \gls{CR}.

Assim, o problema central que esta dissertação procura resolver consiste na ausência de um sistema capaz de apoiar a \gls{CR} na Processware, produzindo comentários automáticos, alinhados com as boas práticas internas, reduzindo esforço repetitivo e contribuindo para um processo mais consistente, eficiente e escalável. Este desafio está alinhado com a necessidade de adaptar \glspl{LLM} a contextos organizacionais específicos, de forma a tirar partido do seu potencial no processo de \gls{CR}~\cite{google2024autocommenter,llmsurvey2023}.

\section{Objetivos}\label{sec:objetivos}

O principal objetivo desta dissertação é desenvolver um sistema capaz de apoiar o processo de \gls{CR} na Processware, fornecendo comentários automáticos consistentes com as boas práticas internas e ajudando a reduzir o esforço associado a tarefas repetitivas. Pretende-se criar uma solução que aumente a eficiência, a uniformidade e a qualidade das revisões de código realizadas na empresa.

Para atingir este propósito, foram definidos os seguintes objetivos:

\begin{enumerate}
    \item \textbf{Identificação e formalização de boas práticas internas:} analisar o código existente e recolher conhecimento junto da equipa responsável pelo processo de \gls{CR} para mapear os padrões, convenções e recomendações atualmente aplicados neste processo.
    \item \textbf{Análise inteligente de alterações de código:} desenvolver um sistema capaz de interpretar \glspl{Diff} e compreender o contexto das alterações submetidas, tendo em conta as várias tecnologias utilizadas na Processware.

    \item \textbf{Geração de comentários consistentes e acionáveis:} criar um mecanismo que produza comentários claros, concisos e alinhados com as boas práticas internas, ajudando os revisores a detetar problemas recorrentes e a manter a consistência entre \glspl{PR}.

    \item \textbf{Integração fluida no fluxo de desenvolvimento:} garantir que a solução pode ser incorporada no processo atual de revisão de código, permitindo que os comentários sejam disponibilizados durante a análise de \glspl{PR}.

    \item \textbf{Avaliação da eficácia e utilidade prática:} validar o sistema através de métricas objetivas e subjetivas, garantindo que a solução contribui para revisões mais eficientes e uniformes.
\end{enumerate}

Este conjunto de objetivos visa assegurar que o sistema desenvolvido traz valor real ao processo de \gls{CR} da Processware, promovendo maior qualidade, consistência e produtividade no desenvolvimento de software.

\section{Solução Proposta}\label{sec:solucao_proposta}

A solução proposta consiste no desenvolvimento de um sistema de apoio à \gls{CR} baseado em \glspl{LLM}, capaz de analisar automaticamente alterações submetidas nas \glspl{PR} e gerar comentários alinhados com as boas práticas internas da Processware. O sistema pretende atuar como um assistente de revisão, reduzindo tarefas repetitivas, promovendo maior consistência entre equipas e aumentando a eficiência do processo.

A proposta assenta nos seguintes componentes principais:

\begin{enumerate}
    \item \textbf{Extração e pré-processamento das alterações de código:} o sistema recebe o \gls{Diff} da \gls{PR} e transforma-o numa representação adequada para análise, garantindo que apenas as alterações relevantes são processadas.
    \item \textbf{Interpretação das mudanças com suporte de \glspl{LLM}:} um modelo de linguagem é utilizado para compreender o contexto das alterações, identificar possíveis problemas e reconhecer padrões que violam boas práticas previamente definidas.

    \item \textbf{Geração de comentários automáticos:} com base na análise realizada, o sistema produz comentários claros, concisos e acionáveis, seguindo as recomendações internas da Processware e facilitando o trabalho dos revisores humanos.

    \item \textbf{Módulo de personalização de boas práticas:} permite adaptar regras e orientações às necessidades da empresa, garantindo que os comentários automáticos refletem padrões internos e preferências específicas de diferentes equipas.

    \item \textbf{Integração com o fluxo de desenvolvimento existente:} o sistema é incorporado diretamente no processo atual da Processware, permitindo que os comentários sejam adicionados nas \glspl{PR} através das ferramentas já utilizadas pela equipa.

    \item \textbf{Monitorização e melhoria contínua:} a solução recolhe métricas e feedback dos utilizadores para identificar oportunidades de melhoria, ajustando o comportamento do sistema de forma progressiva ao longo do tempo.
\end{enumerate}

Em conjunto, estes componentes constituem uma solução prática e extensível, preparada para apoiar revisores humanos e tornar o processo de \gls{CR} mais eficiente, consistente e alinhado com as necessidades reais da empresa.

\section{Estrutura da Dissertação}

O presente documento encontra-se organizado em vários capítulos, que refletem a evolução do trabalho desde o enquadramento teórico até à definição da solução proposta.

O Capítulo 2 apresenta o enquadramento conceptual do processo de revisão de código no contexto do desenvolvimento de software, abordando as suas práticas, objetivos e desafios em ambientes empresariais. São igualmente discutidas as limitações das abordagens tradicionais de revisão e a motivação para a introdução de mecanismos de apoio baseados em inteligência artificial.

O Capítulo 3 analisa as principais tecnologias relevantes para a implementação de sistemas de apoio à revisão de código. São abordadas as plataformas de controlo de versões, as ferramentas tradicionais de análise automática e as soluções baseadas em modelos de linguagem de grande escala (LLMs), bem como as tecnologias associadas à extração, processamento e integração de alterações de código.

O Capítulo 4 apresenta a solução proposta para a integração de inteligência artificial no processo de revisão de código da Processware. Neste capítulo é descrita a abordagem adotada, os requisitos funcionais e não funcionais, a arquitetura geral do sistema e o desenho detalhado dos seus principais componentes, com especial enfoque na integração com os processos internos da organização.

Por fim, os capítulos seguintes (a desenvolver na dissertação final) serão dedicados à implementação da solução proposta, à sua avaliação em contexto real e à análise crítica dos resultados obtidos.

No próximo capítulo serão apresentadas as várias tecnologias a serem utilizadas para o desenvolvimento deste projeto, bem como a análise de trabalhos que sejam considerados relevantes para o enquadramento deste tema de dissertação.
