%!TEX root = ../template.tex
%%%%%%%%%%%%%%%%%%%%%%%%%%%%%%%%%%%%%%%%%%%%%%%%%%%%%%%%%%%%%%%%%%%%
%% chapter2.tex
%% NOVA thesis document file
%%
%% Chapter with the template manual
%%%%%%%%%%%%%%%%%%%%%%%%%%%%%%%%%%%%%%%%%%%%%%%%%%%%%%%%%%%%%%%%%%%%

\typeout{NT FILE chapter2.tex}%

\chapter{Conceitos}\label{cha:conceitos}

Este capítulo tem como objetivo apresentar os principais conceitos teóricos e técnicos que sustentam o desenvolvimento desta dissertação. São abordados os fundamentos relacionados com o processo de \textit{code review} no contexto do desenvolvimento de \textit{software}, bem como a evolução das abordagens de automação aplicadas a este processo.

\section{\textit{Code Review} no Desenvolvimento de \textit{Software}}

A Revisão de Código (\textit{Code Review}) é uma prática fundamental no desenvolvimento de software que consiste na análise sistemática do código-fonte por um ou mais programadores, com o objetivo de identificar defeitos, melhorar a qualidade do código e garantir a conformidade com boas práticas e padrões definidos pela organização~\cite{bacchelli2013expectations,rigby2013convergent}. Esta atividade é normalmente realizada antes da integração das alterações no ramo principal do projeto, assumindo um papel central nos fluxos de desenvolvimento modernos baseados em \textit{pull requests}.

Historicamente, a revisão de código surgiu como uma prática formal, associada a inspeções estruturadas, como as inspeções de Fagan, que demonstraram benefícios significativos na deteção precoce de erros e na redução de custos associados à correção de defeitos em fases avançadas do desenvolvimento~\cite{fagan1976design}. Com a evolução das metodologias ágeis e das plataformas colaborativas de desenvolvimento, a revisão de código tornou-se um processo mais leve e contínuo, integrado no ciclo diário de desenvolvimento de software~\cite{rigby2014peer}.

Os principais objetivos da revisão de código incluem a deteção de erros lógicos e defeitos funcionais, a melhoria da legibilidade e manutenibilidade do código, a verificação da conformidade com padrões de estilo e boas práticas, bem como a mitigação de riscos relacionados com segurança e desempenho~\cite{mcintosh2014empirical}. Para além dos aspetos técnicos, a revisão de código desempenha também um papel relevante na partilha de conhecimento entre membros da equipa, promovendo a disseminação de boas práticas e contribuindo para a uniformização do código ao longo do projeto~\cite{bacchelli2013expectations}.

Apesar dos seus benefícios comprovados, a revisão de código é uma atividade exigente do ponto de vista cognitivo e temporal, dependendo fortemente da experiência e disponibilidade dos revisores. Em projetos de grande escala, caracterizados por elevados volumes de alterações e diversidade tecnológica, este processo pode tornar-se difícil de escalar, originando atrasos, inconsistências e variabilidade na qualidade das revisões realizadas~\cite{rahman2016review}.


\section{Automação da Revisão de Código}

\section{Aprendizagem Automática e \textit{Deep Learning} em \textit{Code Review}}

\section{Modelos de Linguagem de Grande Escala (LLMs)}

\section{\gls{LLMs} aplicados à Revisão de Código}

\section{Limitações atuais e Desafios}
